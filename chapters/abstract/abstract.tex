\chapter{\centering Abstract}
The detection of the $B$-modes polarization anisotropies of the Cosmic
Microwave Background (CMB) polarization pattern at large angular scales
represents a \emph{smoking gun} in favor of the inflactionary paradigm.

Detecting such a faint ($\frac{\Delta T}{T} \leq \num{e-6}$) signal 
requires an accurate study of the systematics and, in the case of ground 
based experiments, atmospheric effects constitute one of the major concern.  

The Strip instrument of the ``Large Scale Polarization Explorer'' (LSPE) is
a ground-based telescope that (from late 2022) will observe the microwave
sky at \SI{43}{\giga\hertz} ($Q$-band) and \SI{95}{\giga\hertz} ($W$-band)
from the ``Observatorio del Teide'' in Tenerife,  attempting to constrain
the $r$ ratio to \num{\approx 0.03}.

In my master thesis I present a novel approach to obtain a statistical
picture of the atmosphere above Pico del Teide, starting from the ERA5
meteorological reanalysis provided by the "Center for Medium-Range Weather
Forecasts" (ECMWF). Secondly, I show the results of the simulations I
performed, by the mean of the software libraries \texttt{CAL} and
\texttt{am}, to evaluate the median annual variations of the atmospheric
brightness temperature above the same site and I compare them with the
measurements acquired by the MFI instrument of the QUIJOTE experiment
(years 2012-2015). Lastly I present the evaluation of the extra white noise
derived from the atmospheric load on the $Q$-band polarimeters of Strip,
obtained performing a bandshape integration of the aforementioned
atmospheric temperatures.
