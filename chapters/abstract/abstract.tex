\section*{\centering Abstract}

Atmospheric emission represents a major concern for ground-based cosmic
microwave background (CMB) experiments. By increasing optical loading on the
detectors, it amplifies their white noise level. In addition, fluctuations
in water vapour content introduce spatial and temporal correlations between
detected signals. I present a novel approach to produce a statistical model
of atmospheric emission from climate reanalysis data and atmospheric
vertical profiles.  This model can be used to estimate the atmospheric
brightness temperature and its seasonal variations at an arbitrary
observation site.  I compare my numerical results with atmospheric
brightness temperatures measured by the QUIJOTE-MFI instrument during the
years 2012-2015, at the ``Observatorio del Teide'' in Tenerife. In
addition, I produce a forecast at \SI{43}{\giga\hertz} for the Q-band of
the LSPE/Strip telescope, which will be deployed to the same site by the end
of 2022.  The atmospheric model is stored in a \num{\sim 2} MB
\texttt{.fits} file and can be easely integrated into simulation pipelines
for CMB experiments.
