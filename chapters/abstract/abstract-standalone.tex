\documentclass[11pt,letterpaper]{article}

\usepackage[a-1a]{pdfx}
\usepackage{hyperref}
\usepackage[T1]{fontenc}
\usepackage{siunitx}

\begin{document}

\section*{\centering Abstract}

My master thesis concerns the formulation and validation of a statistical
model of atmospheric emission in the context of ground-based observations
of the cosmic microwave background radiation (CMB).
\\\\
The cosmic microwave background radiation is a relic signal composed of
photons that were involved in Thomson scattering events occurring in the
primordial plasma until the recombination epoch, about \num{380000} years
after the Big Bang, and have been propagated in the expanding Universe
since then.

The characterization of the polarized component of the CMB signal is one of
the most important goals in observational cosmology. The so called
\emph{B-modes}, the curl component of the CMB polarization field, are
thought to have been generated by gravitational waves propagating in the
primordial Universe. The presence of primordial gravitational waves is one
of the prediction of the inflationary paradigm.

The predicted intensity of the B-modes signal is at the level of fraction
of \si{\micro\kelvin}. Therefore, its detection requires the development of
experiments characterized by extremely high sensitivity. In addition, the
sky signal in the CMB frequency range is dominated by other brighter
radiations, such as galactic foregrounds and atmospheric emission.

Atmospheric effects represents a major concern for ground-based CMB
polarization experiments. Despite being almost unpolarized, atmospheric
emission increases the optical loading on the detectors, amplifying their
white noise level. In addition, its fluctuations, which depend on both the
sky scanning strategy and the properties of the atmosphere at the time of
observations, such as wind speed and water vapour content, introduce
spatial and temporal correlations between detected signals.

The \emph{Large Scale Polarization Explorer} (LSPE) is a next generation
CMB polarization experiment. It has been funded by the \emph{Italian Space
Agency} and aims to provide a smaller upper limit to the intensity of
B-modes. The Strip instrument is the ground-based telescope of the LSPE
project. It will be deployed to the ``Observatorio del Teide'', in
Tenerife, by the end of 2022.

In my thesis I have followed a novel approach to produce a statistical
model of the atmospheric emission, starting from climate reanalysis
provided by the \emph{European Centre for Medium-Range Weather Forecasts}
and atmospheric vertical profiles acquired by balloon probes. This model
can be used to simulate the atmospheric brightness temperature and its
seasonal variations at an arbitrary observation site. It can be stored in a
\num{\sim 2} MB \texttt{.fits} file and can be easily integrated into
simulation pipelines for CMB experiments.

By applying this method to the observing site of LSPE/Strip, I have
produced a forecast of the median daily and annual excursion of the
atmospheric brightness temperature at \SI{43}{\giga\hertz} for the Q-band
polarimeters of the Strip telescope.

In addition, I have compared my numerical results with atmospheric
brightness temperatures measured during the years 2012-2015 by the
\emph{Multi-Frequency Instrument} (MFI) of the QUIJOTE project, which have
been installed to the same site of LSPE/Strip by the ``Instituto de
Astrof\'isica de Canarias''. Through the use of a calibration technique
based on atmospheric vertical profiles acquired on site, I obtain simulated
values of atmospheric brightness temperatures at
\SIlist{11;13;17;19}{\giga\hertz}, which are compatible with MFI data at a
\SI{95}{\percent} confidence level.

The statistical model of atmospheric emission I present in my thesis is
just the starting point to provide a complete representation of the
atmosphere in the microwaves range. More meteorological parameters, such as
wind speed components, can and must be included into the model in order to
take accounts of the atmospheric turbulent structure and the resulting
correlated noise contribution.

\end{document}
