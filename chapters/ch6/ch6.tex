\chapter{Comparison with QUIJOTE-MFI Data}\label{ch:comparison_quijote}

In this chapter values of atmospheric brightness temperature, which have
been obtained making use of the numerical methods described in
\autoref{ch:statistical_picture} and \autoref{ch:atm_variations}, are
compared with measurements acquired by the QUIJOTE experiment.

In particular, the values of $T_\text{atm}$ that have been measured by the
\emph{Multi-Frequency Instrument} (MFI) are considered. MFI is mounted on
the QT1 telescope, which is deployed at the Observatorio del Teide. It
operates since November 2012 in four bands centered at
\SIlist{11;13;17;19}{\giga\hertz}.

\section{QUIJOTE Data and Raw Simulations}

The QUIJOTE dataset is represented in \autoref{fig:quijote_dataset}.
It includes measurements from a total of \num{444} MFI \emph{sky-dip}
observations, performed between December 2012 and February 2015.  On
average, we have one observation every \num{1.78} days. However, the time
distribution of the data is quite inhomogeneous. There are periods in which
we have one observation every sidereal day (sometimes two per day), but
there are some extended periods without observations.

\begin{figure}
        \centering
        \includegraphics[width=\textwidth]{QUIJOTE_Dataset}
        \caption{QUIJOTE-MFI  $T_{atm}$ dataset.}
        \label{fig:quijote_dataset}
\end{figure}

For each point of the scatter plot showed in \autoref{fig:quijote_dataset},
we have used CAL to simulate a value of atmospheric brightness temperature
at the same frequency and at the same hour of a typical day of the
corresponding month. The comparison between simulated data and
$T_\text{atm}$ acquired by the QUIJOTE-MFI instrument is showed in
\autoref{fig:quijote_sim}.

\begin{figure}
        \centering
        \includegraphics[width=\textwidth]{QUIJOTE-Sim}
        \caption{Comparison between QUIJOTE-MFI measurements and CAL
        simulated data.}
        \label{fig:quijote_sim}
\end{figure}

The scatter plot shows that a significant mismatch occurs. In particular
the simulations performed with CAL, making use of ERA5 dataset, yield higher
values of atmospheric brightness temperature. The mismatch becomes
especially problematic for higher frequencies, approaching the
\SI{22}{\giga\hertz} water vapour line. Moreover, QUIJOTE-MFI data
exhibit much greater fluctuations over time.

We recognise that part of the issues showed in \autoref{fig:quijote_sim}
are due to the insufficient spatial resolution of ERA5 reanalysis data.
\autoref{fig:pwv_teide_1980-01-01_12-00} shows a focus on the pixel in
which the Observatorio del Teide is located. In particular we are
interested in the area in which the Strip telescope will be deployed, which
in the figure is signaled by a red circle. The ERA5 dataset only provides a
single value per hour of the PWV for the whole pixel, which constitutes a
spatial average.  Therefore, values of PWV are biased by contributions from
coastal low lands, below \SI{2390}{\meter}, and ocean waters. In other
words, the total column water vapour values that have been taken into
account to evaluate atmospheric brightness temperatures greatly exceeds
true values for Pico del Teide.

\begin{figure}
        \centering
        \includegraphics[width=0.80\textwidth]{PWV_Teide_1980-01-01_12-00}
        \caption{PWV for Pico del Teide.}
        \label{fig:pwv_teide_1980-01-01_12-00}
\end{figure}

\section{The Calibration Coefficient}

To mitigate part of the issues presented in the previous section, we have
employed atmospheric vertical profiles acquired by balloon probes at Pico
del Teide during the year 2018. These vertical profiles reflects the true
weather conditions near the observation site, but are characterized by low
temporal resolution. This is because balloon probes needs approximately
\num{12} hours to measure a whole atmospheric vertical profile. Therefore,
no more than two vertical profiles per day can be acquired.


The free and open source computer program \emph{Atmospheric
Model}\footnote{\url{https://zenodo.org/record/3406483}} (S. Paine) (AM) has
been used to compute sky brightness temperatures starting from the  median
annual vertical atmospheric profile for the year 2018. AM is a tool for
radiative transfer computations at microwave to submillimeter wavelengths.
Spectra which can be computed with AM include thermal emission, absorption,
transmission, and excess delay. Median annual values of $T_\text{sky}$ has
been obtained in the frequency range from \SIrange{10}{50}{\giga\hertz},
with a frequency step of \SI{0.1}{\giga\hertz}.

To compare this result with sky brightness temperatures from CAL,
statistical populations of \num{2736000} elements for the relevant
meteorological parameters have been computedr. As before, we have
initialized the \texttt{Weather} method with the CDFs \texttt{.fits} file
for Pico del Teide. The extracted values of PWV, $T_s$ and $P_s$, which are
showed in \autoref{fig:teide_annual_distributions}, are homogeneously
distributed among the months of the year and the hours of the day.
Expectation values for the meteorological parameters were computed from the
corresponding statistical populations using the same estimator choosen for
balloon vertical profiles. The annual median values and corresponding
standard errors for PWV, $T_s$ and $P_s$ are presented in
\autoref{tab:median_meteo_cal}.  These values have been used as input for
the \texttt{atm\_atmospheric\_loading} function to attain values of $T_\text{sky}$
in the appropriate frequency range.

\begin{figure}
        \centering
        \includegraphics[width=0.93\textwidth]{Teide_Annual_Distributions}
        \caption{Annual distribution for relevant meteorological parameters
        at Pico del Teide.}
        \label{fig:teide_annual_distributions}
\end{figure}

\begin{table}
        \renewcommand{\arraystretch}{1.5}
        \centering
        \begin{tabular}{p{5cm} r}
                \hline
                Parameter & Median  \\
                \hline
                \hline
                PWV \dotfill & \SI{12.3491 \pm 0.0029}{\kilo\gram\per\square\meter} \\
                $T_s$ \dotfill& \SI{289.0434 \pm 0.0025}{\kelvin} \\
                $P_s$ \dotfill &  \SI{92766.55 \pm 0.21}{\pascal} \\
                \noalign{\smallskip}
                \hline
        \end{tabular}
        \caption{CAL median values of relevant meteorological parameters.}
        \label{tab:median_meteo_cal}
\end{table}

The sky brightness temperatures that have been computed making use of CAL
and AM are plotted as a function of $\nu$ in
\autoref{fig:am_cal_comparison}. As expected, the $T_\text{sky}$
from CAL assumes an higher value than that computed with the AM computer
program, when evaluated at the same
frequency. The distance between the two curves becomes particularly
significant near the \SI{22}{\giga\hertz} water vapour absorption line.

\begin{figure}
        \centering
        \includegraphics[width=\textwidth]{AM_CAL_Comparison}
        \caption{AM/CAL sky brightness temperatures comparison.}
        \label{fig:am_cal_comparison}
\end{figure}

We can now introduce a calibration coefficient, $k\qty(\nu)$, to correct
higher values in atmospheric brightness temperature computed by CAL, which
are caused by the excess in total column water vapour in ERA5 data for
the pixel in which the observation site is located. The coefficient is defined as

\begin{equation}
        k_\nu \equiv  k\qty(\nu) \equiv
        \frac{T^\text{AM}_\text{atm}\qty(\nu)}{
        T^\text{CAL}_\text{atm}\qty(\nu)}
\end{equation}

It must be noted that $k_\nu$ is assumed to be time independent. This means
that we think that weather conditions are homogeneous across the pixel, and
non-homogeneity in meteorological parameters at a fixed time only depends
on quantities which are not time dependent, such altitude and distance from
the ocean. The calibration coefficient as a function of frequency is showed
in \autoref{fig:calibration_coefficient_quijote}.

\begin{figure}
        \centering
        \includegraphics[width=\textwidth]{Calibration_Coefficient_QUIJOTE}
        \caption{$K_\nu$ calibration coefficient.}
        \label{fig:calibration_coefficient_quijote}
\end{figure}

The red dots in the figure represent the ratios between the median values of
atmospheric brightness temperatures measured by QUIJOTE-MFI and those of data
simulated with CAL. As it can be seen, they stand in proximity of the
blue line, confirming the validity of the calibration technique at least
for the MFI central frequencies.

\begin{figure}
        \centering
        \includegraphics[width=\textwidth]{QUIJOTE-Sim_calibrated}
        \caption{Comparison between QUIJOTE-MFI measurements and
        CAL simulated data with calibration applied.}
        \label{fig:quijote_sim_calibrated}
\end{figure}

