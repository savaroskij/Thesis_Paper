\begin{otherlanguage}{italian}

\section*{\centering Ringraziamenti}

Il mio primo ringraziamento va al professor Aniello ``Daniele'' Mennella,
senza il quale questo lavoro non avrebbe visto la luce. I suoi consigli e
le sue puntuali osservazioni mi hanno spinto a cercare costantemente di
migliorarmi durante il mio lavoro di tesi. Ringrazio inoltre il dottor
Stefano Mandelli per aver assunto il ruolo di guida sempre presente,
paziente e aperta al confronto e alle critiche. Il suo contributo non si
\`e limitato al solo ambito accademico, ma, in quanto \texttt{algebrato},
egli ha saputo supportarmi con il suo <<Una cosa impossibile alla
volta\dots>> durante i momenti di difficolt\`a, nelle circostanze pi\`u
varie.

Ringrazio i miei genitori, Stefano e Cristina.  Nonostante i loro
abbondanti difetti e limiti (generosamente donati e da me prematuramente
ricevuti in eredit\`a) sono sempre stati un costante punto di riferimento
per me, cos\`i nella vita quotidiana, come nei momenti pi\`u bui.  Sono
stati esempi di tenacia, mantenutasi integra di fronte alle numerose
avversit\`a che la vita pu\`o presentarci, e dispensatori di un costante
flusso di amore non sempre manifestamente accolto.

Ringrazio mia sorella Annalisa. Ho sempre pensato che le nostre teste
fossero state fabbricate da aziende ben distinte, forse concorrenti.  Col
passare del tempo, tuttavia, ho compreso che siamo accumunati dagli stessi
principi e che in lei posso sempre trovare un'alleata sicura. Lei mi ha
insegnato che spesso la virt\`u sta nell'equilibrio, che i giudizi a volte
vanno fatti asciugare al Sole e che c'\`e sempre uno spazio per il dolce.

Ringrazio gli altri componenti della mia famiglia, in particolare i nonni
Cesare, Luisa, Lillo e Carla. Hanno sempre creduto nelle mie capacit\`a con
sorprendente devozione e non mi hanno mai negato una parola gentile, una
poesia, un rammendo, un piatto di spaghetti al sugo o un gelato, un
articolo ritagliato da un giornale o una preghiera in comodato d'uso.
Ringrazio la zia Dada, riportando le sagge parole che ha lasciato qui per
noi: <<Chi troppo studia ignorante diventa, ma chi non studia porta la
brenta>>. Ringrazio le mie zie e zii e le mie ``cuginette''.

Ringrazio le amiche e gli amici incontrati lungo il cammino, anche se
menzionarli tutti \`e un'impresa complessa. Ringrazio in particolare
Poglia, perch\'e in molti momenti, difficili o gioiosi, della mia o della
sua vita, lui era presente e li viveva assieme a me. Le nostre discussioni
sono state portatrici di consiglio nei momenti del dubbio, e di questo
gliene sono grato. Ringrazio Ari per il suo indomabile spirito positivo, il
suo costante volermi bene e il suo <<Hai messo la crema solare?>>. Talvolta
riesco quasi a dimenticare che \`e un ingegnere XD.  Ringrazio la Marti
perch\'e era sempre l\`i anche quando ero proprio KO e per avermi insegnato
che a volte tocca essere easy.  Ringrazio Gloria per il suo sorriso solare, la
sua indispensabile schiettezza e la sua attenzione speciale alle emozioni.
Ringrazio Sara per la sua innata capacit\`a di comprendere e condividere i
miei contorti ragionamenti e stati d'animo.  Ringrazio Anna per avermi
insegnato due o tre utili colpi di pugilato e avermi aiutato a capire che
l'importante \`e non colpirsi da soli.  Ringrazio Sboz perch\'e mi ha
insegnato a non arrendermi mai, ad avere una valutazione quanto pi\`u
oggettiva possibile del mondo e per essere stato un fidato compagno di
avventure, tra il tragico e il comico, fin dalla pi\`u tenera et\`a.
Ringrazio Giangi, Ricky, Bura, Laura, Prenna, Cate, Gabri e tutti le altre
persone con le quali ho condiviso molti momenti felici.

Ringrazio i miei amici e compagni di universit\`a: tra i tanti Dani, Simo,
Emi, Albi, Seba, Vale, Vale a Ale. Questo difficile percorso \`e stato un po' come
quella pubblicit\`a dove quel vaso andava portato in salvo. Solo senza
vaso. Quante dimostrazioni, integrali complessi, quante discussioni al
limite della ragione, sostenute all'interno di mondi pi\`u o meno reali,
quante pizze all'olio piccante, polpette, parry e ``back step'', quanta
musica e quanto altro che \`e meglio non citare in questa sede.

Ringrazio gli amici e i colleghi di LCM e della ``Sez. Reati''. In
particolare \texttt{alessandrodegennaro2}, \texttt{andreatsh},
\texttt{sebastianopagano}, \texttt{martinacrippa},
\texttt{lucacolombogomez}, \texttt{matteozeccolimarazzini},
\texttt{algebrato}, \texttt{sbozzolo}, \texttt{blue}, \texttt{claudiochi},
\texttt{kernel}, \texttt{nanos8}, \texttt{pcteor1}, \texttt{tuc},
\texttt{newborg} e \texttt{born}, \texttt{vcf} e \texttt{videof}, e
\texttt{heisenberg}. Credo che paradossalmente questo luogo polveroso e
ammuffito e i suoi singolari abitanti, a base di silicio o meno, siano
coloro da cui sono stato pi\`u arricchito in termini di conoscenza,
giudizio ed emozioni in tutta la mia carriera universitaria.

Ringrazio i miei insegnanti ed educatori. In particolare, Miriam (e i suoi
tipi di talpe), Il prof. Guzzetti, la prof. Chiarello, la prof.
Colombo, Elena, Francesco e Prenna. Se sono arrivato a scrivere questi
ringraziamenti, il merito \`e sicuramente anche loro.

Un ringraziamento speciale va a tutti i professionisti della neuro di
Garbagnate. In particolare ringrazio la dottoressa Caterina, perch\'e \`e
stata sempre disposta ad ascoltarmi, anche quando tutto ci\`o che esprimevo
erano semplicemente ansie o preccupazioni, mantenendo sempre impresso sul
suo volto un sorriso positivo; Il dottor Marco, per il suo immenso tatto e
la sua chiarezza; le ragazze del MAC, Marianna e Giovanna, per la loro
dolcezza e la loro disponibilit\`a e la dottoressa Laura, che mi ha aiutato
ad imboccare la strada da percorrere per stare bene, anche dentro.  Senza
le loro cure e il loro supporto, manifestato anche oltre la loro
professione, non credo che sarei potuto arrivare fino a qui.

Infine ringrazio anche per le cose brutte e le sfide difficili. Che siano
linfociti ribelli, pandemie globali o altre avventure, hanno dato un sapore
e un significato diverso e speciale alla vita e mi hanno mostrato che in
fondo ho un paio di assi nella manica da tirare fuori al momento giusto.

\end{otherlanguage}
